Detekcia, rozpoznávanie objektov a počítačové videnie všeobecne  je smer IT sféry, ktorý sa veľmi rýchlym krokom rozvíja. Poskytuje ľuďom nové možnosti, ako uľahčiť život, zaistiť bezpečnosť, zabávať sa alebo v rámci podnikovej sféry zrýchliť výrobu, zvýšiť kvalitu výrobkov bez navýšeného úsilia. 

Téma rozpoznávania a detekcie pohybov je aj predmetom tejto práce, v ktorej sa pokúsime postupne popísať konkrétne metódy tohto predmetu. Zároveň sa sústredíme aj na konkrétne riešenie a implementáciu. 

V práci postupne prejdeme od popísania použitých technológií a vývojového prostredia určených na implementáciu k jednotlivým princípom riešenia. Objasníme si, akým spôsobom spracovávať videosekvenciu. Prejdeme si existujúce databázy ľudského pohybu, ich vlastnosti. Ďalej vysvetlíme možnosti a algoritmy spracovávania obrazu, kde detailnejšie popíšeme jednotlivé metódy od prvotných úprav až po klasifikáciu. 

Návrh riešenia odprezentujeme postupne od výberu konkrétnej databázy, cez algoritmus na predspracovanie a spracovanie príznakov, až po klasifikáciu, nástroje na testovanie a samotné porovnanie výsledkov. Na záver zhodnotíme výsledky nášho riešenia a zamyslíme sa nad jeho ďalšími možnosťami rozšírenia a úpravy.
