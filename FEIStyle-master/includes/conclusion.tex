Cieľom našej práce bolo vytvoriť systém, ktorý dokáže detekovať a rozpoznať pohyb osoby zo statického záznamu. 

Zo začiatku sme sa snažili popísať teoretické vlastnosti a princípy rozpoznávania a detekcie ľudského pohybu, jednotlivé metódy a algoritmy, ako aj databázy videí na priblíženie tejto problematiky a našej konkrétnej implementácie.

Na naše riešenie bola zvolená databáza KTH, v ktorej sa nachádzalo 6 typov pohybov. Tieto pohyby sme následne spracovávali pomocou metódy MHI, kde pre jedno video sme využili viacnásobné spracovanie MHI v rôznych časových úsekoch. Ďalším kľúčovým bodom bolo vyňatie prídavných príznakov z MHI pre záznam a spracovanie príznakov pomocou PCA, kde sme taktiež otestovali viacero konfigurácií a možností spracovania. Takto vypočítané príznaky sme následne natrénovali pomocou SVM a testovali úspešnosť. 

Ukázalo sa, že pri viacnásobnom duplikovaní vektorov príznakov nám PCA vyhodnotilo lepšie výsledky ako využitie PCA na pôvodných vektoroch príznakov. Taktiež lepšie výsledky sa prejavili po využití troch MHI z videosekvencie, ktoré boli zachytené v rôznych časových úsekoch videa. 

V konečnej časti sme odprezentovali naše výsledky prostredníctvom tabuliek a percentuálnych úspešností. Tieto výsledky sme porovnali s výsledkami rozpoznávania pomocou iných metód. 
Na záver môžme povedať, že sa nám podarilo vytvoriť rozpoznávací systém, ktorého výsledky sú porovnateľné s inými metódami. Vidíme však priestor na ďalšie vylepšenia a aplikovanie iných a novších metód.

