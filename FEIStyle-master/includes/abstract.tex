Práca sa zaoberá detekciou a rozpoznávaním pohybu na základe informácie z videozáznamu. Hlavný problém spočíva
 v nájdení vhodnej sekvencie metód z oblasti počítačového videnia, ktoré by dokázali čo najpresnejšie rozpoznať pohyb
osoby v praxi. V teoretickej časti práce sú podrobne popísané databázy videí a taktiež podstatné metódy rozpoznávania 
z finálneho riešenia práce. Týmito metódami sú obraz histórie pohybu, histogram orientovaných gradientov, 
analýza hlavných komponentov, metóda podporných vektorov. Implementácia je popísaná postupne od predspracovania
 videa cez výber príznakov až po klasifikáciu a priebežné testy. V záverečnej časti práce sú výsledky porovnané a 
zhodnotené s existujúcim riešením. 